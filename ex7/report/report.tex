\documentclass[10pt]{jarticle}

\usepackage[top=30truemm,bottom=30truemm,left=25truemm,right=25truemm]{geometry}
\usepackage{ascmac}
\usepackage{multicol}

\makeatletter
\def\maketitle{%
  \begin{flushright}
    {\large \@date}
  \end{flushright}
  \begin{center}
    {\LARGE \@title}
  \end{center}
  \begin{flushright}
    {\large \@author}
  \end{flushright}
  \par\vskip 1.5em
}
\makeatother

\title{システムプログラミング演習7}
\author{学籍番号 : 201420694\\
氏名 : 星 遼平}
\date{2015/12/28}

\begin{document}
\maketitle

\section{演習7 - 1}
\begin{itembox}[l]{プログラム}
  \begin{verbatim}
#include <stdlib.h>
#include <string.h>
#include <stdio.h>
#include <pthread.h>

pthread_mutex_t lock = PTHREAD_MUTEX_INITIALIZER;

char *num_str[] = { "one", "two", "three", "four", "five", "six", "seven",
  "eight", "nine", "ten", "eleven", "twelve", "thirteen", "fourteen",
  "fifteen", "sixteen", "seventeen", "eighteen", "nineteen", "twenty",
  "twenty one", "twenty two", "twenty three", "twenty four", "twenty five",
  "twenty six", "twenty seven", "twenty eight", "twenty nine", "thirty" };

int dequeue_count = 0;
int enqueue_count = 0;

struct entry {
  struct entry *next;
  void *data;
};

struct list {
  struct entry *head;
  struct entry **tail;
  pthread_cond_t notempty;
  int size;
};
  \end{verbatim}
\end{itembox}

\begin{itembox}[l]{プログラム(続き)}
  \begin{verbatim}
struct list *list_init(void) {
  struct list *list;

  list = malloc(sizeof *list);
  if (list == NULL)
    return (NULL);
  list->head = NULL;
  list->tail = &list->head;
  list->size = 0;
  return (list);
}

int list_enqueue(struct list *list, void *data) {
  struct entry *e;

  e = malloc(sizeof *e);
  if (e == NULL)
    return (1);
  e->next = NULL;
  e->data = data;
  pthread_mutex_lock(&lock);
  *list->tail = e;
  list->tail = &e->next;
  list->size++;
  pthread_cond_signal(&list->notempty);
  pthread_mutex_unlock(&lock);
  return (0);
}
  \end{verbatim}
\end{itembox}

\begin{itembox}[l]{プログラム(続き)}
  \begin{verbatim}
struct entry *list_dequeue(struct list *list) {
  struct entry *e;

  pthread_mutex_lock(&lock);
  while (list->head == NULL)
    pthread_cond_wait(&list->notempty, &lock);
  if (list->head == NULL) {
    pthread_mutex_unlock(&lock);
    return(NULL);
  }
  e = list->head;
  list->head = e->next;
  list->size--;
  if (list->head == NULL)
    list->tail = &list->head;
  pthread_mutex_unlock(&lock);
  return (e);
}
  \end{verbatim}
\end{itembox}

\begin{itembox}[l]{プログラム(続き)}
  \begin{verbatim}
struct entry *list_traverse
(struct list *list, int (*func)(void *, void *), void *user) {
  struct entry **prev, *n, *next;

  if (list == NULL)
    return (NULL);

  prev = &list->head;
  for (n = list->head; n != NULL; n = next) {
    next = n->next;
    switch (func(n->data, user)) {
      case 0:
        /* continues */
        prev = &n->next;
        break;
      case 1:
        /* delete the entry */
        *prev = next;
        if (next == NULL)
          list->tail = prev;
        return (n);
      case -1:
      default:
        /* traversal stops */
        return (NULL);
    }
  }
  return (NULL);
}

int print_entry(void *e, void *u) {
  printf("%s\n", (char *)e);
  return (0);
}

int delete_entry(void *e, void *u) {
  char *c1 = e, *c2 = u;

  return (!strcmp(c1, c2));
}
  \end{verbatim}
\end{itembox}

\begin{itembox}[l]{プログラム(続き)}
  \begin{verbatim}
// NOTE: 30エントリ追加するための関数
void *put_thirty_entries(void *arg) {
  int i = 0;
  struct list *list = (struct list *)arg;
  for (i = 0; i < 30; i++) {
    enqueue_count++;
    list_enqueue(list, strdup(num_str[i]));
    printf("Enqueue %d回目 : %s\n", enqueue_count, num_str[i]);
  }
}

// NOTE: 10エントリ取り出す関数
void *get_ten_entries(void *arg) {
  struct list *list = (struct list *)arg;
  struct entry *entry;
  int i = 0;

  for (i = 0; i < 10; i++) {
    entry = list_dequeue(list);
    dequeue_count++;
    printf("Dequeue %d回目 : %s\n", dequeue_count, (char *)entry->data);
    free(entry->data);
    free(entry);
  }
}
  \end{verbatim}
\end{itembox}

\begin{itembox}[l]{プログラム(続き)}
  \begin{verbatim}
int main() {
  struct list *list;
  struct entry *entry;

  // NOTE: 30エントリ追加するためのスレッド2つ
  pthread_t put_thirty_entries1, put_thirty_entries2;
  // NOTE: 10エントリ取り出すためのスレッド6つ
  pthread_t get_ten_entries1, get_ten_entries2, get_ten_entries3,
            get_ten_entries4, get_ten_entries5, get_ten_entries6;

  list = list_init();

  pthread_create(&put_thirty_entries1, NULL, put_thirty_entries, (void *)(list));
  pthread_create(&get_ten_entries1, NULL, get_ten_entries, (void *)(list));
  pthread_create(&get_ten_entries2, NULL, get_ten_entries, (void *)(list));
  pthread_create(&get_ten_entries3, NULL, get_ten_entries, (void *)(list));
  pthread_create(&get_ten_entries4, NULL, get_ten_entries, (void *)(list));
  pthread_create(&get_ten_entries5, NULL, get_ten_entries, (void *)(list));
  pthread_create(&put_thirty_entries2, NULL, put_thirty_entries, (void *)(list));
  pthread_create(&get_ten_entries6, NULL, get_ten_entries, (void *)(list));

  pthread_join(put_thirty_entries1, NULL);
  pthread_join(get_ten_entries1, NULL);
  pthread_join(get_ten_entries2, NULL);
  pthread_join(get_ten_entries3, NULL);
  pthread_join(get_ten_entries4, NULL);
  pthread_join(get_ten_entries5, NULL);
  pthread_join(put_thirty_entries2, NULL);
  pthread_join(get_ten_entries6, NULL);

  /* entry list */
  list_traverse(list, print_entry, NULL);

  return (0);
}
  \end{verbatim}
\end{itembox}

\begin{itembox}[l]{実行結果}
  \begin{verbatim}
$ ./app
Enqueue 1回目 : one
Enqueue 2回目 : two
Enqueue 3回目 : three
Enqueue 4回目 : four
Enqueue 5回目 : five
Enqueue 6回目 : six
Enqueue 7回目 : seven
Enqueue 8回目 : eight
Enqueue 9回目 : nine
Enqueue 10回目 : ten
Enqueue 11回目 : eleven
Enqueue 12回目 : twelve
Enqueue 13回目 : thirteen
Enqueue 14回目 : fourteen
Enqueue 15回目 : fifteen
Enqueue 16回目 : sixteen
Enqueue 17回目 : seventeen
Enqueue 18回目 : eighteen
Enqueue 19回目 : nineteen
Enqueue 20回目 : twenty
Enqueue 21回目 : twenty one
Enqueue 22回目 : twenty two
Enqueue 23回目 : twenty three
Enqueue 24回目 : twenty four
Enqueue 25回目 : twenty five
Enqueue 26回目 : twenty six
Enqueue 27回目 : twenty seven
Enqueue 28回目 : twenty eight
Enqueue 29回目 : twenty nine
Enqueue 30回目 : thirty
Dequeue 1回目 : one
Dequeue 4回目 : four
Dequeue 5回目 : five
Dequeue 6回目 : six
Dequeue 7回目 : seven
Dequeue 8回目 : eight
Dequeue 9回目 : nine
Dequeue 10回目 : ten
Dequeue 11回目 : eleven
Dequeue 12回目 : twelve
  \end{verbatim}
\end{itembox}

\begin{itembox}[l]{実行結果(続き)}
  \begin{verbatim}
Dequeue 3回目 : three
Dequeue 13回目 : thirteen
Dequeue 14回目 : fourteen
Dequeue 15回目 : fifteen
Dequeue 16回目 : sixteen
Dequeue 17回目 : seventeen
Dequeue 18回目 : eighteen
Dequeue 19回目 : nineteen
Dequeue 20回目 : twenty
Dequeue 21回目 : twenty one
Dequeue 2回目 : two
Dequeue 22回目 : twenty two
Dequeue 23回目 : twenty three
Dequeue 24回目 : twenty four
Dequeue 25回目 : twenty five
Dequeue 26回目 : twenty six
Dequeue 27回目 : twenty seven
Dequeue 28回目 : twenty eight
Dequeue 29回目 : twenty nine
Dequeue 30回目 : thirty
Dequeue 31回目 : one
Enqueue 31回目 : one
Dequeue 32回目 : two
Enqueue 32回目 : two
Enqueue 33回目 : three
Dequeue 33回目 : three
Dequeue 34回目 : four
Enqueue 34回目 : four
Dequeue 35回目 : five
Enqueue 35回目 : five
Dequeue 36回目 : six
Enqueue 36回目 : six
Dequeue 37回目 : seven
Enqueue 37回目 : seven
Dequeue 38回目 : eight
Enqueue 38回目 : eight
Dequeue 39回目 : nine
Enqueue 39回目 : nine
Dequeue 40回目 : ten
Enqueue 40回目 : ten
Dequeue 41回目 : eleven
Enqueue 41回目 : eleven
  \end{verbatim}
\end{itembox}

\begin{itembox}[l]{実行結果(続き)}
  \begin{verbatim}
Dequeue 42回目 : twelve
Enqueue 42回目 : twelve
Dequeue 43回目 : thirteen
Enqueue 43回目 : thirteen
Dequeue 44回目 : fourteen
Enqueue 44回目 : fourteen
Dequeue 45回目 : fifteen
Enqueue 45回目 : fifteen
Dequeue 46回目 : sixteen
Enqueue 46回目 : sixteen
Dequeue 47回目 : seventeen
Enqueue 47回目 : seventeen
Dequeue 48回目 : eighteen
Enqueue 48回目 : eighteen
Dequeue 49回目 : nineteen
Enqueue 49回目 : nineteen
Dequeue 50回目 : twenty
Enqueue 50回目 : twenty
Dequeue 51回目 : twenty one
Enqueue 51回目 : twenty one
Dequeue 52回目 : twenty two
Enqueue 52回目 : twenty two
Dequeue 53回目 : twenty three
Enqueue 53回目 : twenty three
Dequeue 54回目 : twenty four
Enqueue 54回目 : twenty four
Dequeue 55回目 : twenty five
Enqueue 55回目 : twenty five
Dequeue 56回目 : twenty six
Enqueue 56回目 : twenty six
Dequeue 57回目 : twenty seven
Enqueue 57回目 : twenty seven
Dequeue 58回目 : twenty eight
Enqueue 58回目 : twenty eight
Dequeue 59回目 : twenty nine
Enqueue 59回目 : twenty nine
Dequeue 60回目 : thirty
Enqueue 60回目 : thirty
  \end{verbatim}
\end{itembox}

\newpage

\section{演習7-2}
\begin{itembox}[l]{プログラム}
  \begin{verbatim}
#include <stdlib.h>
#include <string.h>
#include <stdio.h>
#include <pthread.h>

char *num_str[] = { "one", "two", "three", "four", "five", "six", "seven",
  "eight", "nine", "ten", "eleven", "twelve", "thirteen", "fourteen",
  "fifteen", "sixteen", "seventeen", "eighteen", "nineteen", "twenty",
  "twenty one", "twenty two", "twenty three", "twenty four", "twenty five",
  "twenty six", "twenty seven", "twenty eight", "twenty nine", "thirty" };

int dequeue_count = 0;
int enqueue_count = 0;

struct entry {
  struct entry *next;
  void *data;
};

struct list {
  struct entry *head;
  struct entry **tail;
  pthread_mutex_t lock;
  int size;
};

struct list *list_init(void) {
  struct list *list;

  list = malloc(sizeof *list);
  if (list == NULL)
    return (NULL);
  list->head = NULL;
  list->tail = &list->head;
  list->lock = PTHREAD_MUTEX_INITIALIZER;
  list->size = 0;
  return (list);
}
  \end{verbatim}
\end{itembox}

\begin{itembox}[l]{プログラム(続き)}
  \begin{verbatim}
int list_enqueue(struct list *list, void *data) {
  struct entry *e;

  e = malloc(sizeof *e);
  if (e == NULL)
    return (1);
  e->next = NULL;
  e->data = data;
  pthread_mutex_lock(&list->lock);
  *list->tail = e;
  list->tail = &e->next;
  list->size++;
  pthread_mutex_unlock(&list->lock);
  return (0);
}

struct entry *list_dequeue(struct list *list) {
  struct entry *e;

  pthread_mutex_lock(&list->lock);
  if (list->head == NULL) {
    pthread_mutex_unlock(&list->lock);
    return(NULL);
  }
  e = list->head;
  list->head = e->next;
  list->size--;
  if (list->head == NULL)
    list->tail = &list->head;
  pthread_mutex_unlock(&list->lock);
  return (e);
}
  \end{verbatim}
\end{itembox}

\begin{itembox}[l]{プログラム(続き)}
  \begin{verbatim}
struct entry *list_traverse
(struct list *list, int (*func)(void *, void *), void *user) {
  struct entry **prev, *n, *next;

  if (list == NULL)
    return (NULL);

  prev = &list->head;
  for (n = list->head; n != NULL; n = next) {
    next = n->next;
    switch (func(n->data, user)) {
      case 0:
        /* continues */
        prev = &n->next;
        break;
      case 1:
        /* delete the entry */
        *prev = next;
        if (next == NULL)
          list->tail = prev;
        return (n);
      case -1:
      default:
        /* traversal stops */
        return (NULL);
    }
  }
  return (NULL);
}
  \end{verbatim}
\end{itembox}

\begin{itembox}[l]{プログラム(続き)}
  \begin{verbatim}
int print_entry(void *e, void *u) {
  printf("%s\n", (char *)e);
  return (0);
}

int delete_entry(void *e, void *u) {
  char *c1 = e, *c2 = u;

  return (!strcmp(c1, c2));
}

// NOTE: 30エントリ追加するための関数
void *put_thirty_entries(void *arg) {
  int i = 0;
  struct list *list = (struct list *)arg;
  for (i = 0; i < 30; i++) {
    enqueue_count++;
    list_enqueue(list, strdup(num_str[i]));
    printf("Enqueue %d回目 : %s\n", enqueue_count, num_str[i]);
  }
}

// NOTE: 10エントリ取り出す関数
void *get_ten_entries(void *arg) {
  struct list *list = (struct list *)arg;
  struct entry *entry;
  int i = 0;

  for (i = 0; i < 10; i++) {
    entry = list_dequeue(list);
    dequeue_count++;
    printf("Dequeue %d回目 : %s\n", dequeue_count, (char *)entry->data);
    free(entry->data);
    free(entry);
  }
}
  \end{verbatim}
\end{itembox}


\begin{itembox}[l]{プログラム(続き)}
  \begin{verbatim}
int main() {
  struct list *list;
  struct entry *entry;

  // NOTE: 30エントリ追加するためのスレッド2つ
  pthread_t put_thirty_entries1, put_thirty_entries2;
  // NOTE: 10エントリ取り出すためのスレッド6つ
  pthread_t get_ten_entries1, get_ten_entries2, get_ten_entries3,
            get_ten_entries4, get_ten_entries5, get_ten_entries6;

  list = list_init();

  pthread_create(&put_thirty_entries1, NULL, put_thirty_entries, (void *)(list));
  pthread_create(&get_ten_entries1, NULL, get_ten_entries, (void *)(list));
  pthread_create(&get_ten_entries2, NULL, get_ten_entries, (void *)(list));
  pthread_create(&get_ten_entries3, NULL, get_ten_entries, (void *)(list));
  pthread_create(&get_ten_entries4, NULL, get_ten_entries, (void *)(list));
  pthread_create(&get_ten_entries5, NULL, get_ten_entries, (void *)(list));
  pthread_create(&put_thirty_entries2, NULL, put_thirty_entries, (void *)(list));
  pthread_create(&get_ten_entries6, NULL, get_ten_entries, (void *)(list));

  pthread_join(put_thirty_entries1, NULL);
  pthread_join(get_ten_entries1, NULL);
  pthread_join(get_ten_entries2, NULL);
  pthread_join(get_ten_entries3, NULL);
  pthread_join(get_ten_entries4, NULL);
  pthread_join(get_ten_entries5, NULL);
  pthread_join(put_thirty_entries2, NULL);
  pthread_join(get_ten_entries6, NULL);

  /* entry list */
  list_traverse(list, print_entry, NULL);

  return (0);
}
  \end{verbatim}
\end{itembox}

\begin{itembox}[l]{実行結果}
  \begin{verbatim}
$ ./app
Enqueue 1回目 : one
Enqueue 2回目 : two
Enqueue 3回目 : three
Enqueue 4回目 : four
Enqueue 5回目 : five
Enqueue 6回目 : six
Enqueue 7回目 : seven
Enqueue 8回目 : eight
Enqueue 9回目 : nine
Enqueue 10回目 : ten
Enqueue 11回目 : eleven
Enqueue 12回目 : twelve
Enqueue 13回目 : thirteen
Enqueue 14回目 : fourteen
Enqueue 15回目 : fifteen
Enqueue 16回目 : sixteen
Enqueue 17回目 : seventeen
Enqueue 18回目 : eighteen
Enqueue 19回目 : nineteen
Enqueue 20回目 : twenty
Enqueue 21回目 : twenty one
Enqueue 22回目 : twenty two
Enqueue 23回目 : twenty three
Enqueue 24回目 : twenty four
Enqueue 25回目 : twenty five
Enqueue 26回目 : twenty six
Enqueue 27回目 : twenty seven
Enqueue 28回目 : twenty eight
Enqueue 29回目 : twenty nine
Enqueue 30回目 : thirty
Dequeue 1回目 : one
Dequeue 4回目 : four
Dequeue 5回目 : five
Dequeue 6回目 : six
Dequeue 7回目 : seven
Dequeue 8回目 : eight
Dequeue 9回目 : nine
Dequeue 10回目 : ten
Dequeue 11回目 : eleven
Dequeue 12回目 : twelve
  \end{verbatim}
\end{itembox}

\begin{itembox}[l]{実行結果(続き)}
  \begin{verbatim}
Dequeue 3回目 : three
Dequeue 13回目 : thirteen
Dequeue 14回目 : fourteen
Dequeue 15回目 : fifteen
Dequeue 16回目 : sixteen
Dequeue 17回目 : seventeen
Dequeue 18回目 : eighteen
Dequeue 19回目 : nineteen
Dequeue 20回目 : twenty
Dequeue 21回目 : twenty one
Dequeue 2回目 : two
Dequeue 22回目 : twenty two
Dequeue 23回目 : twenty three
Dequeue 24回目 : twenty four
Dequeue 25回目 : twenty five
Dequeue 26回目 : twenty six
Dequeue 27回目 : twenty seven
Dequeue 28回目 : twenty eight
Dequeue 29回目 : twenty nine
Dequeue 30回目 : thirty
Dequeue 31回目 : one
Enqueue 31回目 : one
Dequeue 32回目 : two
Enqueue 32回目 : two
Enqueue 33回目 : three
Dequeue 33回目 : three
Dequeue 34回目 : four
Enqueue 34回目 : four
Dequeue 35回目 : five
Enqueue 35回目 : five
Dequeue 36回目 : six
Enqueue 36回目 : six
Dequeue 37回目 : seven
Enqueue 37回目 : seven
Dequeue 38回目 : eight
Enqueue 38回目 : eight
Dequeue 39回目 : nine
Enqueue 39回目 : nine
Dequeue 40回目 : ten
Enqueue 40回目 : ten
Dequeue 41回目 : eleven
Enqueue 41回目 : eleven
  \end{verbatim}
\end{itembox}

\begin{itembox}[l]{実行結果(続き)}
  \begin{verbatim}
Dequeue 42回目 : twelve
Enqueue 42回目 : twelve
Dequeue 43回目 : thirteen
Enqueue 43回目 : thirteen
Dequeue 44回目 : fourteen
Enqueue 44回目 : fourteen
Dequeue 45回目 : fifteen
Enqueue 45回目 : fifteen
Dequeue 46回目 : sixteen
Enqueue 46回目 : sixteen
Dequeue 47回目 : seventeen
Enqueue 47回目 : seventeen
Dequeue 48回目 : eighteen
Enqueue 48回目 : eighteen
Dequeue 49回目 : nineteen
Enqueue 49回目 : nineteen
Dequeue 50回目 : twenty
Enqueue 50回目 : twenty
Dequeue 51回目 : twenty one
Enqueue 51回目 : twenty one
Dequeue 52回目 : twenty two
Enqueue 52回目 : twenty two
Dequeue 53回目 : twenty three
Enqueue 53回目 : twenty three
Dequeue 54回目 : twenty four
Enqueue 54回目 : twenty four
Dequeue 55回目 : twenty five
Enqueue 55回目 : twenty five
Dequeue 56回目 : twenty six
Enqueue 56回目 : twenty six
Dequeue 57回目 : twenty seven
Enqueue 57回目 : twenty seven
Dequeue 58回目 : twenty eight
Enqueue 58回目 : twenty eight
Dequeue 59回目 : twenty nine
Enqueue 59回目 : twenty nine
Dequeue 60回目 : thirty
Enqueue 60回目 : thirty
  \end{verbatim}
\end{itembox}

\newpage

\section{考察}
スレッドが扱うデータのロック方法は様々あるが、それぞれに特徴があり使いどころ
を間違えないようにしないと、デッドロックの危険性やパフォーマンス低下の
恐れがあるので、各データロックの特徴を把握した上で開発することが重要であるとわかった.

\section{授業の感想}
今回の課題を解く上で授業で説明していた内容がかなり重要になっており、
授業のスライドはあくまで補足説明程度なので、次回が最後ですがメモを取るように
しようと思い直しました。

授業の最初に前回の課題内容の解答・解説がありますが、スライドに含まれていないので
メモを取る時間が少しほしいなと思いました。

\end{document}

