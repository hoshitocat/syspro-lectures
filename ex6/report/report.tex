\documentclass[10pt]{jarticle}

\usepackage[top=30truemm,bottom=30truemm,left=25truemm,right=25truemm]{geometry}
\usepackage{ascmac}
\usepackage{multicol}

\makeatletter
\def\maketitle{%
  \begin{flushright}
    {\large \@date}
  \end{flushright}
  \begin{center}
    {\LARGE \@title}
  \end{center}
  \begin{flushright}
    {\large \@author}
  \end{flushright}
  \par\vskip 1.5em
}
\makeatother

\title{システムプログラミング演習6}
\author{学籍番号 : 201420694\\
氏名 : 星 遼平}
\date{2015/12/21}

\begin{document}
\maketitle

\section{演習6-1}

\begin{itembox}[l]{プログラム}
  \begin{verbatim}
#include <stdlib.h>
#include <stdio.h>
#include <unistd.h>
#include <errno.h>
#include <string.h>
#include <syslog.h>
#include <sys/types.h>
#include <sys/socket.h>
#include <netinet/in.h>

#include "logutil.h"

#define DEFAULT_SERVER_PORT	10000
#ifdef SOMAXCONN
#define LISTEN_BACKLOG SOMAXCONN
#else
#define LISTEN_BACKLOG 5
#endif

char *program_name = "sp6-server";
  \end{verbatim}
\end{itembox}

\begin{itembox}[l]{プログラム(続き)}
  \begin{verbatim}
int open_accepting_socket(int port) {
  struct sockaddr_in self_addr;
  socklen_t self_addr_size;
  int sock, sockopt;

  memset(&self_addr, 0, sizeof(self_addr));
  self_addr.sin_family = AF_INET;
  self_addr.sin_addr.s_addr = INADDR_ANY;
  self_addr.sin_port = htons(port);
  self_addr_size = sizeof(self_addr);
  sock = socket(PF_INET, SOCK_STREAM, 0);
  if (sock < 0)
    logutil_fatal("accepting socket: %d", errno);
  sockopt = 1;
  printf("test\n");
  if (setsockopt(sock, SOL_SOCKET, SO_REUSEADDR,
        &sockopt, sizeof(sockopt)) == -1)
    logutil_warning("SO_REUSEADDR: %d", errno);
  if (bind(sock, (struct sockaddr *)&self_addr, self_addr_size) < 0)
    logutil_fatal("bind accepting socket: %d", errno);
  if (listen(sock, LISTEN_BACKLOG) < 0)
    logutil_fatal("listen: %d", errno);
  return (sock);
}

void usage(void) {
  fprintf(stderr, "Usage: %s [option]\n", program_name);
  fprintf(stderr, "option:\n");
  fprintf(stderr, "\t-d\t\t\t\t... debug mode\n");
  fprintf(stderr, "\t-p <port>\n");
  exit(1);
}
  \end{verbatim}
\end{itembox}

\begin{itembox}[l]{プログラム(続き)}
  \begin{verbatim}
void main_loop(int self_addr) {
  int sock, n;
  struct sockaddr_in client_addr;
  socklen_t client_addr_size;
  char buf[1024];

  client_addr_size = sizeof(client_addr);
  if ((sock =
  accept(self_addr, (struct sockaddr *) &client_addr, &client_addr_size)) < 0) {
    logutil_fatal("deny access: %d", errno);
  }

  while (1) {
    read(sock, buf, 1024);

    if (buf[0] == EOF) {
      logutil_info("disconnected\n");
      break;
    }
    write(sock, buf, strlen(buf) + 1);
  }
  close(sock);
}

int main(int argc, char **argv) {
  char *port_number = NULL;
  int ch, sock, server_port = DEFAULT_SERVER_PORT;
  int debug_mode = 0;

  while ((ch = getopt(argc, argv, "dp:")) != -1) {
    switch (ch) {
      case 'd':
        debug_mode = 1;
        break;
      case 'p':
        port_number = optarg;
        break;
      case '?':
      default:
        usage();
    }
  }
  argc -= optind;
  argv += optind;
  \end{verbatim}
\end{itembox}

\begin{itembox}[l]{プログラム(続き)}
  \begin{verbatim}
  if (port_number != NULL)
    server_port = strtol(port_number, NULL, 0);

  /* server_portでlistenし,socket descriptorをsockに代入 */
  sock = open_accepting_socket(server_port);

  if (!debug_mode) {
    logutil_syslog_open(program_name, LOG_PID, LOG_LOCAL0);
    daemon(0, 0);
  }

  /*
   * 無限ループでsockをacceptし,acceptしたらそのクライアント用
   * のスレッドを作成しプロトコル処理を続ける.
   */
  main_loop(sock);

  /*NOTREACHED*/
  return (0);
}
  \end{verbatim}
\end{itembox}

\begin{itembox}[l]{実行結果}
  \begin{verbatim}
$ ./app -d
test

$ telnet localhost 10000
Trying 127.0.0.1...
Connected to localhost.
Escape character is '^]'.
^CConnection closed by foreign host.

$ ./app -d
test
disconnected
  
  \end{verbatim}
\end{itembox}

\section{演習6-2}

\begin{itembox}[l]{プログラム}
  \begin{verbatim}
#include <stdlib.h>
#include <stdio.h>
#include <unistd.h>
#include <errno.h>
#include <string.h>
#include <syslog.h>
#include <sys/types.h>
#include <sys/socket.h>
#include <netinet/in.h>
#include <signal.h>
#include <pthread.h>

#include "logutil.h"

#define DEFAULT_SERVER_PORT	10000
#ifdef SOMAXCONN
#define LISTEN_BACKLOG SOMAXCONN
#else
#define LISTEN_BACKLOG 5
#endif

char *program_name = "sp6-server";

pthread_mutex_t m = PTHREAD_MUTEX_INITIALIZER;
sigset_t intset;
sigset_t termset;
int sigint = 0;
int sigterm = 0;
  \end{verbatim}
\end{itembox}

\begin{itembox}[l]{プログラム(続き)}
  \begin{verbatim}
int open_accepting_socket(int port) {
  struct sockaddr_in self_addr;
  socklen_t self_addr_size;
  int sock, sockopt;

  memset(&self_addr, 0, sizeof(self_addr));
  self_addr.sin_family = AF_INET;
  self_addr.sin_addr.s_addr = INADDR_ANY;
  self_addr.sin_port = htons(port);
  self_addr_size = sizeof(self_addr);
  sock = socket(PF_INET, SOCK_STREAM, 0);
  if (sock < 0)
    logutil_fatal("accepting socket: %d", errno);
  sockopt = 1;
  printf("test\n");
  if (setsockopt(sock, SOL_SOCKET, SO_REUSEADDR,
        &sockopt, sizeof(sockopt)) == -1)
    logutil_warning("SO_REUSEADDR: %d", errno);
  if (bind(sock, (struct sockaddr *)&self_addr, self_addr_size) < 0)
    logutil_fatal("bind accepting socket: %d", errno);
  if (listen(sock, LISTEN_BACKLOG) < 0)
    logutil_fatal("listen: %d", errno);
  return (sock);
}
  \end{verbatim}
\end{itembox}

\begin{itembox}[l]{プログラム(続き)}
  \begin{verbatim}
void usage(void) {
  fprintf(stderr, "Usage: %s [option]\n", program_name);
  fprintf(stderr, "option:\n");
  fprintf(stderr, "\t-d\t\t\t\t... debug mode\n");
  fprintf(stderr, "\t-p <port>\n");
  exit(1);
}

void *handle_int(void *arg) {
  int sig, err;

  err = sigwait(&intset, &sig);
  if (err || sig != SIGINT)
    abort();
  pthread_mutex_lock(&m);
  logutil_info("\nbye");
  sigint = 1;
  pthread_mutex_unlock(&m);
}

void *handle_term(void *arg) {
  int sig, err;

  err = sigwait(&termset, &sig);
  if (err || sig != SIGTERM)
    abort();
  pthread_mutex_lock(&m);
  logutil_info("bye\n");
  sigterm = 1;
  pthread_mutex_unlock(&m);
}
  \end{verbatim}
\end{itembox}

\begin{itembox}[l]{プログラム(続き)}
  \begin{verbatim}
void main_loop(int self_addr) {
  int sock, n;
  struct sockaddr_in client_addr;
  socklen_t client_addr_size;
  char buf[1024];

  client_addr_size = sizeof(client_addr);
  if ((sock =
    accept(self_addr, (struct sockaddr *) &client_addr, &client_addr_size)) < 0) {
    logutil_fatal("deny access: %d", errno);
  }

  while (1) {
    read(sock, buf, 1024);

    if (buf[0] == EOF) {
      logutil_info("disconnected\n");
      break;
    }

    if (sigint || sigterm) {
      logutil_info("disconnected\n");
      break;
    }

    write(sock, buf, strlen(buf) + 1);
  }
  close(sock);
}
  \end{verbatim}
\end{itembox}

\begin{itembox}[l]{プログラム(続き)}
  \begin{verbatim}
int main(int argc, char **argv) {
  char *port_number = NULL;
  int ch, sock, server_port = DEFAULT_SERVER_PORT;
  int debug_mode = 0;
  pthread_t t;

  while ((ch = getopt(argc, argv, "dp:")) != -1) {
    switch (ch) {
      case 'd':
        debug_mode = 1;
        break;
      case 'p':
        port_number = optarg;
        break;
      case '?':
      default:
        usage();
    }
  }
  argc -= optind;
  argv += optind;
  \end{verbatim}
\end{itembox}

\begin{itembox}[l]{プログラム(続き)}
  \begin{verbatim}
  if (port_number != NULL)
    server_port = strtol(port_number, NULL, 0);

  /* server_portでlistenし,socket descriptorをsockに代入 */
  sock = open_accepting_socket(server_port);

  if (!debug_mode) {
    logutil_syslog_open(program_name, LOG_PID, LOG_LOCAL0);
    daemon(0, 0);
  }

  sigemptyset(&intset);
  sigemptyset(&termset);

  sigaddset(&intset, SIGINT);
  sigaddset(&termset, SIGTERM);

  pthread_sigmask(SIG_BLOCK, &intset, NULL);
  pthread_sigmask(SIG_BLOCK, &termset, NULL);

  pthread_create(&t, NULL, handle_int, NULL);
  pthread_create(&t, NULL, handle_term, NULL);

  /*
   * 無限ループでsockをacceptし,acceptしたらそのクライアント用
   * のスレッドを作成しプロトコル処理を続ける.
   */
  main_loop(sock);

  /*NOTREACHED*/
  return (0);
}
  \end{verbatim}
\end{itembox}

\begin{itembox}[l]{実行結果(SIGINTの場合)}
  \begin{verbatim}
$ ./app -d
test

$ ./app -d
test
^C
bye
  \end{verbatim}
\end{itembox}

\begin{itembox}[l]{実行結果(SIGTERMの場合)}
  \begin{verbatim}
$ ./app -d
test

$ ps aux | grep app
vagrant   5754  0.0  0.0  22888   384 pts/0    Sl+  11:35   0:00 ./app -d

$ ./app -d
test
bye
  \end{verbatim}
\end{itembox}

\section{考察}
シグナルを利用することでCtrl-Cなどの特別な入力やkillコマンドでプロセスを
殺す際にプログラムを作成者が意図したとおりに通ささせることが可能であることが
今回の演習を通して学ぶことができた.シグナルをうまく利用すれば,安全にプロセスを
実行させられる.ただ,killコマンドのオプションで強制的にプロセスを
終了するものがあるが,シグナルをキャッチして強制オプションがつけられても殺せないような
プロセスを実行することが可能であるのかどうか気になった.もし,強制オプションがあっても殺せないような
プロセスを生成してしまったら,少々厄介であると感じた.

\section{授業の感想}
個人的に今回の授業内容は難易度が高いと感じました.また,前回の授業のようにちょっとした
サンプルプログラムを書いて実行してみると理解度も高まったのではないかと感じました.

\end{document}

